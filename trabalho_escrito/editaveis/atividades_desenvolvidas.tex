\chapter{Atividades Desenvolvidas e Cronograma de Execução}

Foram desenvolvidos 2 softwares iniciais de acordo com os objetivos citados. Para o desenvolvimento de ambos os softwares, hoveram atividades em comuns que foram elicitação de requistitos, análise e design de projeto, implementação e implantação.

Com o intuito de prover um ambiente mais flexível para mudança e maturação de requisitos, foi utilizado um método de desenvolvimento empírico, iterativo e incremental. Na engenharia de software se destacam dois processos de controle de desenvolvimento: processo definido e processo empírico. O processo definido é constiuído de um conjunto de sub-processos rigoros nos quais possuem entradas e saídas bem definidas e repetitivas \cite{rup}. Já o processo empírico é constituído de um conjunto de sub-processos imperfeitamente definidos nos quais as entradas e saídas são imprevisíveis e não repetíveis, características essas presentes no desenvolvimento desse trabalho.

A metodologia empírica de desenvolvimento de software se embasa em três fundamentos: precisa ser transparente, visto que o máximo de variáveis devem estar visíveis para os envolvidos no projeto; dado as variáveis expostas a metodologia precisa ser frequentemente inspecionada; feito as inspeções objetivo final é adaptar de acordo com as necessidades. Esses três fundamentos visam ajustar o processo de desenvolvimento para evitar variações de produção inaceitáveis e maximizar a mesma \cite{empirical}.

Para que os procedimentos da metodologia empírica possa ocorrer ela precisa ser de natureza iterativa e incremental. Iterativa e incremental pois terá ciclos curtos de desenvolvimento e a cada ciclo terá incrementos de código. No final de cada ciclo ter-se-à como resultado parâmetros de feedback para a melhoria contínua.

Cada ciclo de desenvolvimento foi trabalhado de forma incremental todas as atividades citadas de tal forma que a cada semana era finalizado uma versão executável de cada software.

No que tange as atividades desenvolvidas para o software $Composer$ foram feitas:
\begin{itemize}
	\item Preparar ambiente e dependências da plataforma Ruby on Rails;
	\item Implementar $laytout$ padrão em HTML, javascript e CSS;
	\item Implementar coluna no banco de dados para referenciação de imagens;
	\item Implementar método para leitura de arquivos de áudio do tipo .WAV;
	\item Implementar método para processamento de áudio do tipo .WAV;
	\item Implementar método para $download$ de áudio do tipo .WAV.
\end{itemize}

No que tange as atividades desenvolvidas para o software $Snake$ $Chords$ foram feitas:
\begin{itemize}
	\item Preparar ambiente e dependências da plataforma Android;
	\item Estudar e implementar exemplo de jogo disponível no pacote SDK-Android;
	\item Implementar classe de reconhecimento de acordes musicais em Java;
	\item Adaptação da classe de reconhecimento de acordes para os recursos e métodos específicos da plataforma Android;
	\item Integrar classe de reconhecimento de acordes musicais ao jogo;
	\item Implementar design interativo e gráfico do jogo.
\end{itemize}

Segue cronograma das atividades desenvolvidas:
\begin{table}[h]
\centering
\caption{Cronograma das Atividades}
\label{my-label}
\begin{tabular}{|l|l|l|l|l|l|l|l|l|l|l|l|l|}
\hline
Atividade/Semana & 1 & 2 & 3 & 4 & 5 & 6 & 7 & 8 & 9 & 10 & 11 & 12 \\ \hline
Composer - 1     & x &   &   &   &   &   &   &   &   &    &    &    \\ \hline
Composer - 2     & x & x &   &   &   &   &   &   &   &    &    &    \\ \hline
Composer - 3     &   & x & x &   &   &   &   &   &   &    &    &    \\ \hline
Composer - 4     &   &   & x & x &   &   &   &   &   &    &    &    \\ \hline
Composer - 5     &   &   &   & x & x &   &   &   &   &    &    &    \\ \hline
Composer - 6     &   &   &   & x & x & x &   &   &   &    &    &    \\ \hline
Snake Chords - 1 &   &   &   &   & x & x &   &   &   &    &    &    \\ \hline
Snake Chords - 2 &   &   &   &   &   & x &   &   &   &    &    &    \\ \hline
Snake Chords - 3 &   &   &   &   &   &   & x &   &   &    &    &    \\ \hline
Snake Chords - 4 &   &   &   &   &   &   &   & x &   &    &    &    \\ \hline
Snake Chords - 5 &   &   &   &   &   &   &   &   & x & x  & x  &    \\ \hline
Snake Chords - 6 &   &   &   &   &   &   &   &   &   &    &    & x  \\ \hline
\end{tabular}
\end{table}
