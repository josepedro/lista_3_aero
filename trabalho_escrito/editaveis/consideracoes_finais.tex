\chapter{Considerações Finais}

Em vista do que foi desenvolvido ao longo do estágio, houve vários desafios a serem vencidos que o ambiente controlado das disciplinas da graduação são limitadas em sanar. O primeiro deles é o que tange o processo de negócio da instituição abordada. Tal processo revelou-se bastante específico pois a demanda foi de aplicações que iam além do trandicional $CRUD$. Dado que as técnicas tradicionais de engenharia de software mapeiam muito bem o processo de negócio no processo de desenvolvimento no que diz respeito a software baseados em consulta, armazenamento, exclusão e atualização de dados, houve várias adaptações dessas mesmas técnicas para software multimídias trabalhos no estágio.

Outro ponto importante a ser considerado foi a dificuldade em comunicar e transparecer aspectos para leigos envolvidos nos projetos. Tais falhas de comunicação tiveram que ser trabalhadas rigorosamente e de forma transparente para que os produtos estivessem claros e objetivos para os envolvidos no projeto. O melhoramento da comunicação se refletiu bastante na consolidação de requisitos para implementação do software.

Por fim, o aprendizado principal é que, para se desenvolver um software razoável num dado período de projeto, é preciso que todos desenvolvedores tenham contato com toda a cadeia de produção do mesmo, desde modelagem de processos de negócio até implantação do sistema dentro de um ambiente computacional com recursos limitados. Tal consciência integral da cadeia de desenvolvimento contribui diretamente para a qualidade do código em si, atingindo, assim, as reais necessadidades da organização.