\chapter{Descrição da Empresa/Instituição}
MídiaLab Laboratório de Pesquisa em Arte Computacional Este espaço de arte e pesquisa, foi criado em 1986 com o nome de Laboratório de Imagem e Som, em 2000 passou a ser denominado de Laboratório de Pesquisa em Arte e realidade Virtual e atualmente, em função da abrangência das pesquisas realizadas intitula-se de MídiaLab Laboratório de pesquisa em arte computacional.
Ele é coordenado por Suzete Venturelli e conta com a participação de bolsistas de Iniciação Científica. estagiários e estudantes da graduação e do programa de pós-graduação em Arte, linha de pesquisa em arte e tecnologia, que trabalham em diferentes propostas, envolvendo a criação de animação, vídeos, arte computacional, dispositivos não convencionais de interação, ciberintervenções urbanas, Realidade aumentada urbana (RUA), entre outros. Os projetos envolvem questões socio-artísicas e políticas no contexto da arte, ciência e tecnologia realizados em estreita colaboração (parceria, consórcio ou prestação de serviços) com outras áreas de pesquisa como a ciência da computação, mecatrônica, esporte, saúde e comunicação, para propor projetos inovadores, artísticos e tecnologicamente interessantes para se pensar a sociedade hoje.
Procura-se deste modo concretizar as sinergias existentes entre o tecido criativo, o ensino superior e a investigação realizada na Universidade de Brasília, proporcionando um contexto de formação avançada e promovendo a criação de projetos economicamente sustentáveis.
As atividades do MídiaLab exploram a Estética, as Tecnologias da Informação e Comunicação nas seguintes vertentes \footnote{Para análise mais detalhada da instituição conferir em http://www.midialab.unb.br}:
\begin{itemize}
\item Concepção, desenvolvimento, integração e operação de soluções artísticas, educativas e tecnocientíficas;
\item Parcerias nacionais e internacionais em projetos de arte computacional;
\item Promoção do empreendedorismo e apoio a produção cultural de base tecnocientífica;
\item Formação especializada (projetos, estágios e seminários de divulgação artística e tecnocientífica);
\item Apoio a propostas no âmbito de programas nacionais e comunitários;
\item Prospecção artística, tecnológica e inovação;
\item Consultoria e assessoria.
\end{itemize}
