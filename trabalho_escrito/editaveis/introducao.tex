\pdfbookmark[0]{\contentsname}{toc}
\tableofcontents*
\cleardoublepage

\chapter{Lista de Exercícios 3}

\section{Questão 1 - Forneça o valor calculado da pressão acústica.}
Para o cálculo da pressão acústica utilizou-se a seguinte sequência de processamento:
\begin{enumerate}
	\item Os dados de velocidades das matrizes Ux, Uy e Uz foram importados nas variáveis
``velocidades.vel\_x'', ``velocidades.vel\_y'' e ``velocidades.vel\_z'' respectivamente;
	\item Valores físicos para densidade inicial, comprimento elementar e posição do ouvinte no espaço foram definidos;
	\item A função ``calcular\_pressao()'' foi chamada para o calculo da pressao;
	\item Dentro da função ``calcular\_pressao()'' foram dados os seguintes processos:
	\begin{enumerate} 
		\item Foi calculado o valor de \begin{math}vivj\end{math} através soma de todas as velocidades elevadas ao quadrado (simplificação do Tensor de Lighthill);
		\item \begin{math}vivj\end{math} foi multiplicado pela densidade inicial;
		\item Foi construída uma matriz unitária de dimensões 100 X 95 X 100 no intuito de preenchê-la com o escalar calculado anteriormente;
		\item Foi calculado a subtração da posição do observador por cada ponto da região de turbulência;
		\item Derivou-se, para cada direção, 2 vezes, objetivando calcular o laplaciano do Tensor de Lighthill;
		\item Integrou-se o resultado através de integral de volume com o método numérico trapeziodal.
	\end{enumerate}


\end{enumerate}

Segue o código do $script$ principal:
	\begin{lstlisting}
	clear('all');
	close all;

	% Lista de Exercicios 3

	disp('Questao 1.1 ----------------------');
	velocidades = open('velocidades.mat');
	velocidades_x = velocidades.vel_x(:,:,1);
	velocidades_y = velocidades.vel_y(:,:,1);
	rho = 1.2; % kg/m^3
	delta_x = 0.003; % m
	posicao_ouvinte = [15 15 15]; % m

	pressao_acustica = calcular_pressao(rho, delta_x, ...
	velocidades.vel_x, velocidades.vel_y, ...
	posicao_ouvinte);

	valor_referencia = 2*10^-5;
	nivel_pressao_sonora_dB = 20*log10((pressao_acustica ...
	+valor_referencia)/valor_referencia);

	resposta_1 = ['Valor de Pressao Acustica: ', ... 
	num2str(pressao_acustica, '%10.5e'), ...
	' N/m^2'];
	disp(resposta_1);

	resposta_2 = ['Valor de Nivel de Pressao Sonora: ' ... 
	, num2str(nivel_pressao_sonora_dB), ...
	' dB'];
	disp(resposta_2);

	disp('Questao 1.2 ----------------------');
	% Plotando mapa de superficie de velocidade absoluta
	velocidades_absolutas = sqrt(velocidades_x.^2 + ...
	velocidades_y.^2);
	[x,y] = meshgrid([0:99]*0.003);
	x = x(:, 1:95);
	y = y(:, 1:95);
	figure;
	surf(velocidades_absolutas);
	title('Grafico de Velocidades Absolutas');
	xlabel('x');
	ylabel('y');
	zlabel('velocidade absoluta [m/s]');
	vorticidade = curl(velocidades_x, velocidades_y);
	figure;
	surf(x, y, vorticidade);
	title('Grafico de Velocidades Vorticiais');
	ylabel('y');
	xlabel('x');
	zlabel('velocidade vorticial [rad/s]');
	resposta = ['O valor da dimensao caracteristica eh 0.0630 metros ou 63 mm', ...
	' pois a turbulencia possui um centro definido com esse diametro caracteristico.'];
	disp(resposta);	

	\end{lstlisting}

	\section{Questão 2 - }

	\begin{lstlisting}
	clear('all');
	close all;

	% Lista de Exercicios 3

	disp('Questao 1.2 ----------------------');
	% Plotando mapa de superficie de velocidade absoluta
	velocidades_absolutas = sqrt(velocidades_x.^2 + ...
	velocidades_y.^2);
	[x,y] = meshgrid([0:99]*0.003);
	x = x(:, 1:95);
	y = y(:, 1:95);
	figure;
	surf(velocidades_absolutas);
	title('Grafico de Velocidades Absolutas');
	xlabel('x');
	ylabel('y');
	zlabel('velocidade absoluta [m/s]');
	vorticidade = curl(velocidades_x, velocidades_y);
	figure;
	surf(x, y, vorticidade);
	title('Grafico de Velocidades Vorticiais');
	ylabel('y');
	xlabel('x');
	zlabel('velocidade vorticial [rad/s]');
	resposta = ['O valor da dimensao caracteristica eh 0.0630 metros ou 63 mm', ...
	' pois a turbulencia possui um centro definido com esse diametro caracteristico.'];
	disp(resposta);	
	\end{lstlisting}

	\section{Questão 3 - }

	\begin{lstlisting}
	clear('all');
	close all;

	% Lista de Exercicios 3

	disp('Questao 1.3 ----------------------');
	dimensao_caracteristica_l = 0.063; % m
	distancia = sqrt(sum(posicao_ouvinte.^2)); % m 
	c0 = 340; % m/s 
	velocidade_inicial = ((pressao_acustica*(distancia)*c0^2)/(dimensao_caracteristica_l*rho))^(1/4);
	% Plotando grafico de pressao por velocidades atraves da equacao 1
	pressao_velocidades_1(1:10) = 0;
	pressao_velocidades_2(1:10) = 0;
	velocidades_media_x = velocidades.vel_x;
	tamanhos = size(velocidades.vel_x);
	velocidade_media_x = (sum(sum(sum(velocidades_x))))/tamanhos(1)*tamanhos(2)*tamanhos(3);
	velocidades_media_x(:) = 1;
	velocidades_media_x = velocidades_media_x*velocidade_media_x;
	velocidades_media_y = velocidades.vel_y;
	tamanhos = size(velocidades.vel_y);
	velocidade_media_y = (sum(sum(sum(velocidades_y))))/tamanhos(1)*tamanhos(2)*tamanhos(3);
	velocidades_media_y(:) = 1;
	velocidades_media_y = velocidades_media_y*velocidade_media_y;
	for divisao = 1:10
		velocidade_divisao = 10^divisao;
		pressao_velocidades_1(divisao) = calcular_pressao(rho, delta_x, velocidades_media_x/velocidade_divisao ...
		, velocidades_media_y/velocidade_divisao, posicao_ouvinte);
		velocidade = velocidade_inicial/velocidade_divisao;
		pressao_velocidades_2(divisao) = (dimensao_caracteristica_l/distancia)*(rho*velocidade^4)/c0^2;
	end
	figure;
	loglog(pressao_velocidades_1);
	hold on;
	loglog(pressao_velocidades_2, 'r');
	title('Grafico Pressao Sonora em Relacao a Velocidade Absoluta');
	ylabel('pressao acustica');
	xlabel('velocidade');
	legend('Equacao de Lighthill', 'Aproximacao da Oitava Potencia');
	\end{lstlisting}

 Pergunta: os resultados coincidem? Justifique a sua resposta de maneira crítica.
 Os dois gráficos possuem comportamentos similares visto que a pressão sonora decai exponencialmente ao longo
 da variação de velocidades. Esse fato ocorre pois nas duas equações se considera que o som é gerado a partir 
 de somas compactas oriundas de fontes sonoras, independentes,
 que possuem o volume definido por V0/l\^3, dado que l é a dimensão característica de cada vórtice.
 Dado esse contexto, na expansão em campo distante o termo de retardamento se aproxima de 0 pois é considerado
 que a análise dos vórtices é feita na origem do sistema, desconsiderando assim o efeito do retardamento. Nesse caso
 a integral da solução de Green em campo distante se delimita em pho0*v\^2*l\^3.